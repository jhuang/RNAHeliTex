\documentclass[ignorenonframetext,10pt]{beamer}
%\documentclass[paper=a4,12pt,version=last,landscape]{scrartcl}

%\usepackage[latin1]{inputenc}
\usepackage[english]{babel}
\usepackage{amsmath,amssymb}
\usepackage{multimedia}
\usepackage{alltt}
\usepackage{multirow}
\usepackage{textcomp}
\usepackage[footnotesize]{subfigure}
\usepackage{graphicx}


\usepackage{pgfplots}
\pgfplotsset{width=7cm,compat=1.3}% <-- moves axis labels near ticklabels (respects tick label widths)
\usepackage{pgfplotstable}



%\usetheme{Berlin}
\usetheme{Darmstadt}
\useoutertheme[subsection=false]{miniframes}
\useoutertheme{smoothbars}
\usefonttheme{structurebold}
%\setbeamertemplate{navigation symbols}{}
\setbeamercovered{invisible}


\title{Helices of RNAs}
\author{\large Jiabin~Huang}
\date{\today}

\institute[ExpBI]{\normalsize
  AG Experimentelle Bioinformatik (Cyanolab)\\
  Institut f\"ur Biologie III\\
  Universit\"at Freiburg}
\subject{Talk based on Giegerich, Voss, Rehmsmeier (2004) ''Abstract
Shapes of RNA'' Nucleic Acids Research Vol. 32 No. 16}
  
  


\begin{document}

\frame{\maketitle}

% frame 2
\begin{frame}
\frametitle{Outline}
   \begin{itemize}
   \item RNA secondary structure components and classical algorithms  
   \item Introducing concept of abstract shapes  
   \item Development of a new structure abstraction 
   \item Outlook            
   \end{itemize}
\end{frame}


  
\begin{frame}
\frametitle{Secondary structure components of RNA}  
\begin{figure}
  \includegraphics[scale=0.4]{images/RNA_components.jpg} 
  \caption{Structural components}
\end{figure}
\end{frame}



\section{RNA folding by free energy minimization}
\subsection{}
\begin{frame}
\frametitle{Classical secondary structure algorithms}
  \begin{block}{Simplifying Assumptions for Structure Prediction}
    \begin{itemize}
    \item RNA folds into one minimum free-energy structure.
    \item There are no knots (base pairs never cross).
    \item The energy of a particular base pair in a double stranded regions is
    sequence independent
    \end{itemize} 
  \end{block}
  \begin{block}{RNA has different structural components}
    \begin{itemize}
    \item single-stranded regions (first described by Zuker and Stiegler in
    1981)
    \item RNA folding are stabilized by the stacking or by dangling end
    \item destabilized by hairpin, internal and bulge loops
    \item the runtime is $O(n^3)$
    \end{itemize}
  \end{block}
\end{frame}


% frame 4
\begin{frame}
\frametitle{Free energy computation example}  
\begin{figure}
  \includegraphics[scale=0.4]{images/mfe_example.pdf} 
\end{figure}
\end{frame}


% frame 5
\begin{frame}
\frametitle{energy landscape}
\begin{figure}
  \includegraphics[scale=0.4]{images/energy_landscape_simple.jpg}
\end{figure}
\end{frame}


% frame 6
\begin{frame}
\frametitle{Suboptimal structures}
  %\begin{block}
    But the ''true'' structure is not always the one with the lowest
    predicted free energy.
    So what to do?
    \begin{itemize} 
    \item Enumerate suboptimal structures within a given energy range R.
    \item Hope to find a structure fulfilling your expectation or coming close
    to experimental results.
    \end{itemize}
    But the number of suboptimal structures grows exponentially with the energy
    range considered.
  %\end{block}   
\end{frame}


% frame 7
\begin{frame}
\frametitle{energy landscape}  
\begin{figure}
  \includegraphics[scale=0.5]{images/Yeast.png} 
\end{figure}
\end{frame}


% frame 6
\begin{frame}
\frametitle{Introducing abstract shapes}
    Solution: Use abstract shapes to describe a set of structures.
    \begin{itemize} 
    \item developed by Giegerich and Voss
    \item An abstract shape represents a class of similar structures sharing a
    common pattern of helix nesting and adjacency.
    \item ''Abstract'' since we do not care about all details of the structures.
    \item Each shape class has a representative structure called shrep (with minimum folding energy).
    \end{itemize}
\end{frame}



% frame 9
\begin{frame}
\frametitle{Abstract shape, Energy range: 5 kcal/mol}  
\begin{figure}
  \includegraphics[scale=0.6]{images/shapes_example.jpg} 
\end{figure}
\end{frame}


% frame 10
\section{develop a new structure abstraction}
\subsection{}
\begin{frame}
\frametitle{}
   \begin{block}{\small Drawback of abstract shape}
   \begin{itemize} 
   \item The major drawback of abstract shape analysis is the position
   independence of the abstraction
   \end{itemize}
   \end{block}
   \begin{block}{\small Drawback of abstract shape}
   \begin{figure}
     \includegraphics[scale=0.55]{images/drawback_1.jpg} 
   \end{figure}
   \end{block}
   \begin{block}{\small Drawback of abstract shape}
   \begin{figure}
     \includegraphics[scale=0.55]{images/drawback_2.jpg} 
   \end{figure}
   \end{block}
\end{frame}



\begin{frame}
\frametitle{develop a new structure abstraction}
    The straightforward idea to overcome the position independence of the
    current available shape abstractions is
    \begin{block}{what keeps track of positions}
    \begin{itemize} 
    \item helical regions of hairpin loop
    \item helical regions of multiloops %, as possible braching points, are structurally important    
    \item helical regions of stacked pairs, bulge and internal loops %are the main structural contributors 
    \end{itemize}
    \end{block}
    \begin{block}{which positions of the base pair will be tracked}
    \begin{itemize} 
    \item i
    \item j   
    \item (i,j)
    \item i+j/2
    \end{itemize}    
    \end{block}
\end{frame}

% frame landscape
\begin{frame}
\frametitle{energy landscape}  
\begin{figure}
  \includegraphics[scale=0.4]{images/helices_position.jpg} 
\end{figure}
\end{frame}

% frame 11
\begin{frame}[fragile]
  Output from the first trial version of helices shape
  \begin{block}{\small Helices shape, Energy range: 5 kcal/mol}
  %\begin{alltt}
  \begin{verbatim}
        UCGCGCACAGGACAUCCUAGGUACAAGGCCGCCGU
-6.30   .((.((..(((....))).(((.....))))))).  [13.5,25]  
-4.60   ........(((....))).(((........)))..  [13.5,26.5] 
-3.90   ....((..(((....)))..)).............  [13.5]
-3.60   .........((....(((.......)))...))..  [22]
-3.40   ....((..(((....)))..))....((...))..  [13.5,30]
-3.20   ..((((.....((.......)).....)).))...  [17]
-2.80   .........((........(((.....))).))..  [25]
-2.40   ...................(((........)))..  [26.5]
-1.60   ....((..(((....)))..)).....((....))  [13.5,31.5]
  \end{verbatim} 
  %\end{alltt}
  \end{block}
\end{frame}


% frame landscape
\begin{frame}
\frametitle{energy landscape}  
\begin{figure}
  \includegraphics[scale=0.4]{images/energy_landscape.jpg} 
\end{figure}
\end{frame}


\begin{frame}
\frametitle{Possible problems}
    \begin{itemize} 
    \item do not preserve nesting of elements and might lead to abstract shapes
    where a helix index occurs more than once $\rightarrow$ solved with
    different representation form
    \item abstracting from bulge and internal loops might be to rigorous
    $\rightarrow$ refine the definition of the abstraction, such that all critical criteria
    are met
    \item records of helices is a lot more than the records of shapes
    $\rightarrow$ runtime problem  %translation of ..的多
    \end{itemize}
\end{frame}

\begin{frame}
\frametitle{Growth of structure, helix and shape space}
    \centering
	\begin{tikzpicture}
		%\begin{semilogyaxis}[xlabel=Index,ylabel=Value]
		%\addplot gnuplot[color=blue,mark=*]{1.2196*(x**(-1.5))*(2.6180**x)}; 
		%\end{semilogyaxis}
		\begin{semilogyaxis}[legend style={font=\small},xlabel={Sequence
         length},ylabel={Nr. of
         Structures/Helices/Shapes},width=\textwidth,legend style={nodes=right},
         legend pos= north west]
     
        
         \addplot+[only marks,mark=+]
         table[x=X,y=Y]{/home/jhuang/workspace/RNAHeliCes/scripts/estimate_exponent_RNAsubopt.txt};
         \addlegendentry{Structures}         
         \addplot table[x=X,y={create col/linear
         regression={y=Y}}]{/home/jhuang/workspace/RNAHeliCes/scripts/estimate_exponent_RNAsubopt.txt};
         \addlegendentry{$0.33 \cdot x -3.24$}
         
         
         \addplot+[only marks,mark=-]
         table[x=X,y=Y]{/home/jhuang/workspace/RNAHeliCes/scripts/estimate_exponent_RNAhelix.txt};
         \addlegendentry{Helices}         
         \addplot table[x=X,y={create col/linear
         regression={y=Y}}]{/home/jhuang/workspace/RNAHeliCes/scripts/estimate_exponent_RNAhelix.txt};
         \addlegendentry{$0.18 \cdot x -1.42$} 
 
         \addplot+[only marks,mark=x]
         table[x=X,y=Y]{/home/jhuang/workspace/RNAHeliCes/scripts/estimate_exponent_RNAshapes_5.txt};
         \addlegendentry{Shapes}                   
         \addplot table[x=X,y={create col/linear
         regression={y=Y}}]{/home/jhuang/workspace/RNAHeliCes/scripts/estimate_exponent_RNAshapes_5.txt};                 
         \addlegendentry{
$\pgfmathprintnumber{\pgfplotstableregressiona} \cdot x
\pgfmathprintnumber[print sign]{\pgfplotstableregressionb}$}
         
               
        
        %\addlegendentry{Legenden Eintrag};
        \end{semilogyaxis} 

	\end{tikzpicture}
\end{frame}


\begin{frame}
\frametitle{Growth of helix space  }
    \centering
	\begin{tikzpicture}
		%\begin{semilogyaxis}[xlabel=Index,ylabel=Value]
		%\addplot gnuplot[color=blue,mark=*]{1.2196*(x**(-1.5))*(2.6180**x)}; 
		%\end{semilogyaxis}
		\begin{semilogyaxis}[legend style={font=\tiny},xlabel={Sequence
         length},ylabel={Nr. of Helices},width=\textwidth,legend
         style={nodes=right}, legend pos= north west]

         \addplot+[only marks,mark=-]
         table[x=X,y=Y]{/home/jhuang/workspace/RNAHeliCes/scripts/estimate_exponent_RNAhelix.txt};
         \addlegendentry{energy range unlimited}
         \addplot table[x=X,y={create col/linear
         regression={y=Y}}]{/home/jhuang/workspace/RNAHeliCes/scripts/estimate_exponent_RNAhelix.txt};
         \addlegendentry{$0.18 \cdot x -1.42$} 

          \addplot+[only marks,mark=x]
          table[x=X,y=Y]{/home/jhuang/workspace/RNAHeliCes/scripts/estimate_exponent_RNAhelix_5_20kcal.txt};
          \addlegendentry{energy range 20kcal/mol}
          \addplot table[x=X,y={create col/linear
          regression={y=Y}}]{/home/jhuang/workspace/RNAHeliCes/scripts/estimate_exponent_RNAhelix_5_20kcal.txt}; 
          \addlegendentry{$0.12 \cdot x +0.44$} 
       
          \addplot+[only marks,mark=x]
          table[x=X,y=Y]{/home/jhuang/workspace/RNAHeliCes/scripts/estimate_exponent_RNAhelix_5_15kcal.txt};
          \addlegendentry{energy range 15kcal/mol}
          \addplot table[x=X,y={create col/linear
          regression={y=Y}}]{/home/jhuang/workspace/RNAHeliCes/scripts/estimate_exponent_RNAhelix_5_15kcal.txt};
          \addlegendentry{$0.0904 \cdot x +1.23$} 

          \addplot+[only marks,mark=x]
          table[x=X,y=Y]{/home/jhuang/workspace/RNAHeliCes/scripts/estimate_exponent_RNAhelix_5_10kcal.txt};
          \addlegendentry{energy range 10kcal/mol}
          \addplot table[x=X,y={create col/linear
          regression={y=Y}}]{/home/jhuang/workspace/RNAHeliCes/scripts/estimate_exponent_RNAhelix_5_10kcal.txt};
          \addlegendentry{$0.0602 \cdot x +1.43$}      
                                        
          \addplot+[only marks,mark=x]
          table[x=X,y=Y]{/home/jhuang/workspace/RNAHeliCes/scripts/estimate_exponent_RNAhelix_5_5kcal.txt};
          \addlegendentry{energy range 5kcal/mol}
          \addplot table[x=X,y={create col/linear
          regression={y=Y}}]{/home/jhuang/workspace/RNAHeliCes/scripts/estimate_exponent_RNAhelix_5_5kcal.txt};
          \addlegendentry{$0.0366 \cdot x +0.7$} 
     
                            
%           \addlegendentry{
%  $\pgfmathprintnumber{\pgfplotstableregressiona} \cdot x
%  \pgfmathprintnumber[print sign]{\pgfplotstableregressionb}$}
         
               
        
        %\addlegendentry{Legenden Eintrag};
        \end{semilogyaxis} 

	\end{tikzpicture}
\end{frame}


\begin{frame}
\frametitle{Outlook: Designing RNA class predictors}
    \begin{itemize} 
    \item develop abstraction
    \item evaluate abstraction
    \item algorithm implementation
    \item algorithm evaluation
    \item class predictors
    \end{itemize}
\end{frame}

\begin{frame}
\frametitle{End}
   \begin{itemize} 
   \item Thanks a lot for your attention !
   \item Questions ?
   \end{itemize}
\end{frame}


\end{document}
